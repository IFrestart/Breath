\documentstyle[12pt,a4]{article}

\begin{document}

\author{Jori Liesenborgs (jori@lumumba.luc.ac.be)}
\title{JRTPLIB (v2.7) - TUTORIAL}
\date{10th December 2002}
\maketitle

\section{Getting started}

\subsection{Creating a session}

To be able to do anything useful with the library, you'll have to create a RTP
session. To do this, you have to create a variable of type RTPSession and call
its Create member. The function 'Create' takes one parameter which specifies
the portbase for the session. So a simple (and totally useless) program would
look something like this:
    \begin{verbatim}

    #include "rtpsession.h"
        
    int main(void)
    {
            RTPSession sess;
        
        sess.Create(5000);
        return 0;
    }
    \end{verbatim}

\subsection{Errors}

What if something goes wrong and the session can't be created ? Well, in the
library, all functions returning an 'int' return a negative value if an error
occurred, so you can detect if a function was successfull or not. Now, an
'int' is probably not the most understandable error message and that's why the
library has the function RTPGetErrorString. This takes the return value of a
function as an argument and returns a pointer to a string which describes the
error. We could change the previous program to write out what happened:
    \begin{verbatim}
    #include <stdio.h>
    #include "rtpsession.h"
        
    int main(void)
    {
        RTPSession sess;
        int status;
        char *msg;
            
        status = sess.Create(5000);
        msg = RTPGetErrorString(status);
        printf("%s\n",msg);
        return 0;
    }
    \end{verbatim}

\subsection{Other initialization}

One thing you'll probably need to initialize is the timestamp unit, since this
value is needed for various calculations. The timestamp unit can be set through
the use of the RTPSession member function SetTimestampUnit. This function takes
one argument: the timestamp unit in seconds.
    
    
    For example, if you're using the session to send 8000Hz sampled audio,
you'd normally use the number of samples as RTP timestamp. Since the timestamp
would then increment by 8000 each second, the timestamp unit would be 1/8000
seconds. This is the default value which is used in the library. For 44100Hz
sampling, the code would look something like this:
        \begin{verbatim}
        sess.SetTimestampUnit(1.0/44100.0);
        \end{verbatim}
        
\section{Sending and receiving data}

\subsection{Specifying destinations}

Before you can send data, you must specify to which destinations each packet
should be sent. You can specify this using the RTPSession member functions
AddDestination, DeleteDestination and ClearDestinations. For example:
        \begin{verbatim}
        unsigned long addr = ntohl(inet\_addr("127.0.0.1"));
        sess.AddDestination(addr,5000);
        \end{verbatim}

would prepare the session to send data to the local host, to port 5000. Note
that the 'ntohl' function had to be used because 'inet\_addr' returns an
address in network byte order and the parameters functions has to be in host
byte order.

\subsection{Sending RTP packets}

When you have entered the destinations for the RTP packets, you can start to
send data to all destinations by using the SendPacket member function of
RTPSession. This is an overloaded member function. In one version you have
to pass the RTP payload type, the 'mark' flag, and the timestamp increment
as parameters. Here's an example:
        \begin{verbatim}
        sess.SendPacket("1234567890",10,0,false,10);
        \end{verbatim}
The first parameter is the data which has to be sent, the second is the length
of the data. Next come the payload type and the 'mark' flag. Finally, the last
argument specifies the amount by which the timestamp should be incremented.


If you often need the same payload type, 'mark' flag and timestamp increment,
you can use the second version of the function. To use it, you must first
specify default values for those parameters. This can be done with the
RTPSession member functions SetDefaultPayloadType, SetDefaultMark and
SetDefaultTimeStampIncrement. Then, you can use the SendPacket function without
the last three arguments. For example, once you've entered the default values
        \begin{verbatim}
        sess.SetDefaultPayloadType(0);
        sess.SetDefaultMark(false);
        sess.SetDefaultTimeStampIncrement(10);
        \end{verbatim}
you can do exactly what the previous example of SendPacket did, by simply doing
this:
        \begin{verbatim}
        sess.SendPacket("1234567890",10);
        \end{verbatim}

\subsection{Receiving RTP packets}

First of all, you should call the RTPSession member function PollData on a
regular basis. This function processes incoming RTP and RTCP packets. To
access the received RTP packets, you'll typically iterate over the different
participants in the session and get each one's RTP packets. You can iterate
over the participants using the functions GotoFirstSource and GotoNextSource.
If you're only interested in the participants which have data you didn't get
yet, you can use GotoFirstSourceWithData and GotoNextSourceWithData. All of
these functions return true if there's a source available and false if not.
When a source is available, you can use the function GetNextPacket to get a
RTP packet from this source. When you don't need the RTP packet anymore, make
sure you delete it. Here's some example code:
        \begin{verbatim}
        if (sess.GotoFirstSourceWithData())
        {
            do
            {
                RTPPacket *pack;
                
                pack = sess.GetNextPacket();
                
                // process packet
                
                delete pack;
                
            } while (sess.GotoNextSourceWithData());
        }
        \end{verbatim}

For a description about the RTPPacket class, you should take a look at the
reference manual (manual.txt).

\subsection{Receive modes}

There are three receive modes. A receive mode determines which incoming
packets are accepted and which are ignored. You can set the receive mode by
using the RTPSession member function SetReceiveMode.
    
    
    The first mode, RECEIVEMODE\_ALL, is the default. When you've enabled
this mode simply all incoming packets are accepted.
    
    
    The second mode, RECEIVEMODE\_IGNORESOME, accepts all incoming packets
except the ones which are coming from certain senders. You can determine which
packets are ignored by using AddToIgnoreList, DeleteFromIgnoreList and
ClearIgnoreList.
    
    
    The last mode, RECEIVEMODE\_ACCEPTSOME, ignores all incoming packets
except the ones which are coming from certain senders. You can specify which
packets are accepted by using AddToAcceptList, DeleteFromAcceptList and
ClearAcceptList.


The AddTo... and DeleteFrom... functions take three arguments. The first one
is the IP address of the involved sender. The second one is a flag. If set to
true, all packets originating from the previous IP address are ignored or
accepted and the third parameter is ignored. If set to false, the third
parameter specifies a port number. When this is the case, only packets with
the specified IP-port combination are ignored or accepted.
    
    
    An important note is that the port which has to be specified is the
port from which the RTP packets originate and this does not have to be the RTP
portbase. More important, in this library those ports are not the same ! The
port from which packets are sent can be retrieved by the function GetSendPort.

\subsection{Multicasting}

\subsubsection{Sending to a multicast group}

To be able to send to a multicast group you don't have to do anything special.
You can simply add the multicast address for that group to the list of
destinations and RTP and RTCP data will automatically be sent to it.
    
    
    The only thing that probably needs to be set is the Time To Live (TTL)
field for multicast packets. This is a field in the IP header which controls
how far (how many hops) a packet can travel. The value is initialized to one
so normally a multicast packet will not leave the local network. If you do
require this, you can alter the value by using the RTPSession member function
SetMulticastTTL.

\subsubsection{Receicing from a multicast group}

You must first join a multicast group to be able to receive packets which are
sent to it. You can control from which multicast groups you want to receive
data by using the functions JoinMulticastGroup, LeaveMulticastGroup and
LeaveAllMulticastGroups. The first two of these functions take a multicast
IP address as an argument.

\subsection{About sending and receiving RTCP packets}

You don't have to worry about sending and receiving RTCP packets. The library
takes care of this. The only thing you have to make sure is that the PollData
is called on a regular basis. During a poll, the library reads incoming RTCP
packets and processes the information. It also checks if it's time to send
RTCP packets and if so, it sends them. This check is also performed when you
call the SendPacket function.

\section{Control information}

\subsection{Own information}

You can set your local information by using the following member functions of
RTPSession: SetLocalName, SetLocalEMail, SetLocalLocation, SetLocalPhone,
SetLocalTool and SetLocalNote. All of these functions take two parameters: the
first one is a 'char *' to the data, the second one specifies how many bytes
of the data should be used. For example,
        \begin{verbatim}
        rtpsess.SetLocalEMail("user@myhost.org",15);
        \end{verbatim}
could be used to set the local E-mail address.


You can specify which information should be sent by using the following
RTPSession member functions: EnableSendName, EnableSendEMail,
EnableSendLocation, EnableSendPhone, EnableSendTool and EnableSendNote. All of
these functions take a boolean as argument. This argument specifies whether or
not to send certain information about the current session.

\subsection{Other sessions' information}

To retrieve the information about other participating RTP sessions, you have
to specify which source's information you want. You can do this by iterating
over the sources using the GotoFirst... and GotoNext... functions, as was
described in section 2.3. Then, you can do a call to the RTPSession member
function GetCurrentSourceInfo. This function returns a data structure of type
RTPSourceData. You can use this to look at the available information about
the source.
    
    
    There is another way to retrieve information about a source. To be
able to use it, you already have to know the sychronization source identifier
(SSRC) of the source. You can then call the RTPSession member function
GetSourceInfo. This function takes a SSRC as argument and returns a data
structure of type RTPSourceData.
    
    
    For more information about the RTPSourceData class and all of its
member functions, you should take a look at the reference manual, which
can be found in the file 'manual.txt'.

\section{Handlers}

You can specify handlers for certain types of events. For example: when a
source joins the session, when a source times out, when SSRC identifiers
collide, ...
    
    
    For more information about the use of handlers you should take a look
at the reference manual (manual.txt).

\end{document}
